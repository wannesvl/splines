\documentclass{standalone}

\usepackage{fontspec}
\setmainfont[BoldFont=DTLProkyonSTMedium,ItalicFont=DTLProkyonSTLightItalic, Numbers={Lining}]{DTLProkyonSTLight}
\let\sfdefault\rmdefault

\usepackage{tikz,enumerate}
\usepackage{verbatim, bm}
\usepackage[active,tightpage]{preview}

\usetikzlibrary{arrows,automata,decorations.markings}
\usetikzlibrary{positioning}
\usetikzlibrary{calc,backgrounds,fadings}

\renewcommand{\labelitemi}{$\bullet$}
\renewcommand{\labelitemii}{\textbf{-}}

\definecolor{red}{rgb}{0.75,0,0}
\newcommand{\techTitle}[1]{\textcolor{red}{\textbf{\Large #1}}}%
\newcommand{\techSubtitle}[1]{\textcolor{red}{\textbf{ #1}}}%

\definecolor{blue}{rgb}{0,0,0.5}
\newcommand{\mpTitle}[1]{\textcolor{blue}{\textbf{\Large #1}}}%
\newcommand{\mpSubtitle}[1]{\textcolor{blue}{\textbf{ #1}}}%

\definecolor{green}{rgb}{0,0.5,0}
\newcommand{\robTitle}[1]{\textcolor{green}{\textbf{\Large #1}}}%
\newcommand{\robSubtitle}[1]{\textcolor{green}{\textbf{ #1}}}%

\newcommand{\mprobTitle}[1]{\textcolor{blue!50!green}{\textbf{\Large #1}}}%
\newcommand{\mprobSubtitle}[1]{\textcolor{blue!50!green}{\textbf{ #1}}}%

\tikzset{
    thick/.style=      {line width=1.2pt},
    very thick/.style= {line width=1.6pt},
    ultra thick/.style={line width=5pt}
}


\begin{document}

\begin{preview}
\begin{tikzpicture}[>=latex]

\tikzstyle{tech} = [rectangle, rounded corners,
                    fill = black!10,
                    draw = red, thick,
                    minimum height = 13cm, minimum width = 15cm]

\tikzstyle{app1} = [rectangle, rounded corners,
                    fill = black!10,
                    draw = blue, thick,
                    minimum height = 10cm, minimum width = 10cm]

\tikzstyle{app2} = [rectangle, rounded corners,
                    fill = black!10,
                    draw = green, thick,
                    minimum height = 10cm, minimum width = 10cm]

\tikzstyle{app12} = [rectangle, rounded corners,
                     fill = black!10,
                     draw = blue!50!green, thick,
                     minimum height = 10cm, minimum width = 10cm]



%%%%%%%%%%%%%%%%%%%%%%%%%%%%%%%%%%%%%%%%%%%%%%%%%%%%%%%%%%%%%%%%%%%%%%%%%%%%
%%%%%%%%%%%%%%%%%%%%%%%%%%%%%%%%%%%%%%%%%%%%%%%%%%%%%%%%%%%%%%%%%%%%%%%%%%%%

\node[draw=red, fill=red!20, rounded corners, very thick, anchor=north, minimum height=16cm, minimum width=35cm] at (0,2) (tech){};%
\node at ($(tech.north)+(0,-0.75)$) {\textbf{\LARGE \textcolor{red}{ENABLING TECHNOLOGIES}}};%


%%%%%%%%%%%%%%%%%%%%%%%%%%%%%%%%%%%%%%%%%%%%%%%%%%%%%%%%%%%%%%%%%%%%%%%%%%%%
%%%%%%%%%%%%%%%%%%%%%%%%%%%%%%%%%%%%%%%%%%%%%%%%%%%%%%%%%%%%%%%%%%%%%%%%%%%%

\node[tech, anchor=north] at($(tech.north)+(9,-2)$) (Bprog){};%
\node[anchor=north] at (Bprog.north) {
\begin{tabular}{c}
\mbox{~}\\[-6pt]
\techTitle{Spline Optimization}\\[12pt]
\parbox{14cm}{

\techSubtitle{What?}
\begin{itemize}\vspace*{-4pt} \itemsep 0pt
\item optimization problems with tensor-product spline variables
\item semi-infinite spline constraints are relaxed to constraints on the spline coefficients
\end{itemize}

\techSubtitle{Software}
\begin{itemize}\vspace*{-4pt} \itemsep 0pt
\item Yalmip based Matlab toolbox
\color{gray}
\item Python toolbox
\color{black}
\end{itemize}

\techSubtitle{Open issues: methodology}
\begin{itemize}\vspace*{-4pt} \itemsep 0pt
\item dual interpretation of the relaxation
\item positivity on non-hyperrectangular domains
    \begin{itemize} \vspace*{-4pt} \itemsep 0pt
    \item simplex splines (for polyhedral domains only)
    \item tensor-product spline mapping of hyperrectangle to parameter domain \\(relation with simplex splines for polyhedral domains?)
    \item relation with Polya's relaxation for convex polyhedrons
    \end{itemize}
\item dealing with complex variables (e.g. in unit disc or RHP)
    \begin{itemize}\vspace*{-4pt} \itemsep 0pt
    \item one variable instead of real and imaginary part
    \item  trigonometric splines (when properly redefined)
    \end{itemize}
\end{itemize}

\techSubtitle{Open issues: software}
\begin{itemize}\vspace*{-4pt} \itemsep 0pt
\item numerical stability and efficiency, particularly for multi-variate splines
\item exploiting symmetry and structure, eliminating redundant constraints \ldots
\end{itemize}

}\end{tabular}};


%%%%%%%%%%%%%%%%%%%%%%%%%%%%%%%%%%%%%%%%%%%%%%%%%%%%%%%%%%%%%%%%%%%%%%%%%%%%
%%%%%%%%%%%%%%%%%%%%%%%%%%%%%%%%%%%%%%%%%%%%%%%%%%%%%%%%%%%%%%%%%%%%%%%%%%%%

\node[tech, anchor=north] at ($(tech.north)+(-9,-2)$) (CCprog){};%
\node[anchor=north] at (CCprog.north) {
\begin{tabular}{c}
\mbox{~}\\[-6pt]
\techTitle{Convex Concave Programming}\\[12pt]
\parbox{14cm}{

\techSubtitle{What?}
\begin{itemize}\vspace*{-4pt} \itemsep 0pt
\item nonlinear optimization problems with inequality constraints decomposable in a convex and a concave part
\end{itemize}

\techSubtitle{Software}
\begin{itemize}\vspace*{-4pt} \itemsep 0pt
\item SCP
\item SCLB
\end{itemize}

\techSubtitle{Open issues: methodology}
\begin{itemize}\vspace*{-4pt} \itemsep 0pt
\item best way to alternate between outer and inner approximations
\end{itemize}

\techSubtitle{Open issues: software}
\begin{itemize}\vspace*{-4pt} \itemsep 0pt
\item BMI
\end{itemize}

}\end{tabular}};


%%%%%%%%%%%%%%%%%%%%%%%%%%%%%%%%%%%%%%%%%%%%%%%%%%%%%%%%%%%%%%%%%%%%%%%%%%%%
%%%%%%%%%%%%%%%%%%%%%%%%%%%%%%%%%%%%%%%%%%%%%%%%%%%%%%%%%%%%%%%%%%%%%%%%%%%%

\node[draw=blue, fill=blue!20, opacity=0.5, rounded corners, very thick, anchor=north, minimum height=24.5cm, minimum width=33cm] at ($(tech.south)+(-11,-3)$) (mot){};%
\node at ($(mot.north)+(0,-0.75)$) {\textbf{\LARGE \textcolor{blue}{OPTIMAL MOTION PLANNING}}};%


%%%%%%%%%%%%%%%%%%%%%%%%%%%%%%%%%%%%%%%%%%%%%%%%%%%%%%%%%%%%%%%%%%%%%%%%%%%%
%%%%%%%%%%%%%%%%%%%%%%%%%%%%%%%%%%%%%%%%%%%%%%%%%%%%%%%%%%%%%%%%%%%%%%%%%%%%

\node[draw=green, fill=green!20, opacity=0.5, rounded corners, very thick, anchor=north, minimum height=24.5cm, minimum width=44cm] at ($(tech.south)+(16.5,-4)$) (rob){};%
\node at ($(rob.north)+(0,-0.75)$) {\textbf{\LARGE \textcolor{green}{ROBUST OPTIMIZATION}}};%


%%%%%%%%%%%%%%%%%%%%%%%%%%%%%%%%%%%%%%%%%%%%%%%%%%%%%%%%%%%%%%%%%%%%%%%%%%%%
%%%%%%%%%%%%%%%%%%%%%%%%%%%%%%%%%%%%%%%%%%%%%%%%%%%%%%%%%%%%%%%%%%%%%%%%%%%%

\node[app12, anchor=north] at($(mot.north)+(11,-2.5)$) (flat){};%
\node[anchor=north] at (flat.north) {%
\begin{tabular}{c}
\mbox{~}\\[-6pt]
\mprobTitle{Optimal Control of Flat Systems}\\[12pt]
\parbox{9cm}{

\mprobSubtitle{What?}
\begin{itemize}\vspace*{-4pt} \itemsep 0pt
\item linear \& (polynomially) nonlinear differentially flat systems
\item flat output parameterized as spline; state and input constraints amount to semi-infinite spline constraints
\item both path following and fee motion
\end{itemize}

\mprobSubtitle{Software}
\begin{itemize}\vspace*{-4pt} \itemsep 0pt
\item Python ??? switch to matlab toolbox
\end{itemize}

\mprobSubtitle{Open issues: methodology}
\begin{itemize}\vspace*{-4pt} \itemsep 0pt
\item anti-collision constraints
    \begin{itemize}\vspace*{-4pt} \itemsep 0pt
    \item convex combination of collision free outer paths
    \item tensor-product spline mapping of hyperrectangle to nonconvex feasible set
    \end{itemize}
\end{itemize}

\mprobSubtitle{Open issues: software}
\begin{itemize}\vspace*{-4pt} \itemsep 0pt
\item real-time implementation
\end{itemize}

} \end{tabular} };


%%%%%%%%%%%%%%%%%%%%%%%%%%%%%%%%%%%%%%%%%%%%%%%%%%%%%%%%%%%%%%%%%%%%%%%%%%%%
%%%%%%%%%%%%%%%%%%%%%%%%%%%%%%%%%%%%%%%%%%%%%%%%%%%%%%%%%%%%%%%%%%%%%%%%%%%%

\node[app2, anchor=north] at($(rob.north)+(-5.5,-2)$) (par){};%
\node[anchor=north] at (par.north) {%
\begin{tabular}{c}
\mbox{~}\\[-6pt]
\robTitle{Parametric Programming}\\[12pt]
\parbox{9cm}{

\robSubtitle{What?}
\begin{itemize}\vspace*{-4pt} \itemsep 0pt
\item polynomial spline parameterized approximate optimizer functions
\item feasibility for all parameter values amount to semi-infinite spline constraints
\end{itemize}

\robSubtitle{Software}
\begin{itemize}\vspace*{-4pt} \itemsep 0pt
\item Matlab spline-opt toolbox
\end{itemize}

\robSubtitle{Open issues: methodology}
\begin{itemize}\vspace*{-4pt} \itemsep 0pt
\item ???
\end{itemize}

\robSubtitle{Open issues: software}
\begin{itemize}\vspace*{-4pt} \itemsep 0pt
\item ???
\end{itemize}

} \end{tabular} };

%%%%%%%%%%%%%%%%%%%%%%%%%%%%%%%%%%%%%%%%%%%%%%%%%%%%%%%%%%%%%%%%%%%%%%%%%%%%

\node[app2, anchor=north] at($(par.south)+(0,-1.5)$) (parapp){};%
\node[anchor=north] at (parapp.north) {%
\begin{tabular}{c}
\mbox{~}\\[-6pt]
\robTitle{Applications}\\[12pt]
\parbox{9cm}{

\robSubtitle{Time-optimal point-to-point motion}
\begin{itemize}\vspace*{-4pt} \itemsep 0pt
\item practical evaluation
\item application to combined structure control design
\end{itemize}

\robSubtitle{Explicit MPC}
\begin{itemize}\vspace*{-4pt} \itemsep 0pt
\item practical evaluation
\item analysis/proof of convergence
\end{itemize}

\robSubtitle{Trade-off curves}
\begin{itemize}\vspace*{-4pt} \itemsep 0pt
\item application to (combined structure control) design
\end{itemize}

\robSubtitle{Nonconvex programs with few non-convexifying variables}
\begin{itemize}\vspace*{-4pt} \itemsep 0pt
\item ???
\end{itemize}

\robSubtitle{LPV control}
\begin{itemize}\vspace*{-4pt} \itemsep 0pt
\item better averaged performance with slightly worse worst-case performance?
\end{itemize}

} \end{tabular} };

%%%%%%%%%%%%%%%%%%%%%%%%%%%%%%%%%%%%%%%%%%%%%%%%%%%%%%%%%%%%%%%%%%%%%%%%%%%%

\draw[ultra thick, black!40, ->] ($(par.south)$) -- ($(parapp.north)$);%


%%%%%%%%%%%%%%%%%%%%%%%%%%%%%%%%%%%%%%%%%%%%%%%%%%%%%%%%%%%%%%%%%%%%%%%%%%%%
%%%%%%%%%%%%%%%%%%%%%%%%%%%%%%%%%%%%%%%%%%%%%%%%%%%%%%%%%%%%%%%%%%%%%%%%%%%%

\node[app2, anchor=north] at($(rob.north)+(5.5,-2)$) (nlp){};%
\node[anchor=north] at (nlp.north) {%
\begin{tabular}{c}
\mbox{~}\\[-6pt]
\robTitle{Nonconvex Programming}\\[12pt]
\parbox{9cm}{

\robSubtitle{What?}
\begin{itemize}\vspace*{-4pt} \itemsep 0pt
\item hierarchy of spline-based relaxations for global optimization of nonconvex programs
\end{itemize}

\robSubtitle{Software}
\begin{itemize}\vspace*{-4pt} \itemsep 0pt
\item Matlab spline-opt toolbox
\end{itemize}

\robSubtitle{Open issues: methodology}
\begin{itemize}\vspace*{-4pt} \itemsep 0pt
\item class of optimization problems
\item asymptotic exactness
\item determining exactness / indication of conservatism from dual
\end{itemize}

\robSubtitle{Open issues: software}
\begin{itemize}\vspace*{-4pt} \itemsep 0pt
\item ???
\end{itemize}

} \end{tabular} };


%%%%%%%%%%%%%%%%%%%%%%%%%%%%%%%%%%%%%%%%%%%%%%%%%%%%%%%%%%%%%%%%%%%%%%%%%%%%
%%%%%%%%%%%%%%%%%%%%%%%%%%%%%%%%%%%%%%%%%%%%%%%%%%%%%%%%%%%%%%%%%%%%%%%%%%%%

\node[app2, anchor=north] at($(rob.north)+(16.5,-2)$) (sys){};%
\node[anchor=north] at (sys.north) {%
\begin{tabular}{c}
\mbox{~}\\[-6pt]
\robTitle{Linear Systems and Control}\\[12pt]
\parbox{9cm}{

\robSubtitle{What?}
\begin{itemize}\vspace*{-4pt} \itemsep 0pt
\item analysis and control of linear systems: (uncertain) LTI, LPV, interconnected systems \ldots based on well-posedness
\end{itemize}

\robSubtitle{Software}
\begin{itemize}\vspace*{-4pt} \itemsep 0pt
\item ???
\end{itemize}

\robSubtitle{Open issues: methodology}
\begin{itemize}\vspace*{-4pt} \itemsep 0pt
\item full-full block multipliers
    \begin{itemize}\vspace*{-4pt} \itemsep 0pt
    \item reducing conservatism in synthesis
    \item interpretation from the frequency dependent multiplier viewpoint
    \end{itemize}
\end{itemize}

\robSubtitle{Open issues: software}
\begin{itemize}\vspace*{-4pt} \itemsep 0pt
\item ???
\end{itemize}

} \end{tabular} };

%%%%%%%%%%%%%%%%%%%%%%%%%%%%%%%%%%%%%%%%%%%%%%%%%%%%%%%%%%%%%%%%%%%%%%%%%%%%

\node[app2, anchor=north] at($(sys.south)+(0,-1.5)$) (sysapp){};%
\node[anchor=north] at (sysapp.north) {%
\begin{tabular}{c}
\mbox{~}\\[-6pt]
\robTitle{Applications}\\[12pt]
\parbox{9cm}{

\robSubtitle{Structured controller design}
\begin{itemize}\vspace*{-4pt} \itemsep 0pt
\item approximate convex approach
\item solving BMIs
\end{itemize}

\robSubtitle{Design of metamaterials}
\begin{itemize}\vspace*{-4pt} \itemsep 0pt
\item improving numerical conditioning
\item reducing computational load
\end{itemize}

} \end{tabular} };

%%%%%%%%%%%%%%%%%%%%%%%%%%%%%%%%%%%%%%%%%%%%%%%%%%%%%%%%%%%%%%%%%%%%%%%%%%%%

\draw[ultra thick, black!40, ->] ($(sys.south)$) -- ($(sysapp.north)$);%


%%%%%%%%%%%%%%%%%%%%%%%%%%%%%%%%%%%%%%%%%%%%%%%%%%%%%%%%%%%%%%%%%%%%%%%%%%%%
%%%%%%%%%%%%%%%%%%%%%%%%%%%%%%%%%%%%%%%%%%%%%%%%%%%%%%%%%%%%%%%%%%%%%%%%%%%%

\draw[ultra thick, black!40, ->] ($(Bprog.south)$) .. controls ($(Bprog.south)+(0,-2)$) and ($(rob.north)+(0,2)$) .. ($(rob.north)$)%
node[black,midway] {\large\textbf{spline based relaxations}};%

\draw[ultra thick, black!40, ->] ($(CCprog.south)$) .. controls ($(CCprog.south)+(0,-4)$) and ($(sys.north)+(0,4)$) .. ($(sys.north)$)%
node[black,near start] {\large\textbf{solving structured control problems}};%

\draw[ultra thick, black!40, ->] ($(CCprog.south)$) .. controls ($(CCprog.south)+(0,-2)$) and ($(mot.north)+(0,2)$) .. ($(mot.north)$)%
node[black,midway] {\large\textbf{solving some motion planning problems}};%

% add arrow rob -> mot (anti-collision)

\end{tikzpicture}
\end{preview}
\end{document}
