
\documentclass{standalone}

\usepackage{tikz}
\usepackage{verbatim, bm}
\usepackage[active,tightpage]{preview}

\usetikzlibrary{arrows,automata}
\usetikzlibrary{positioning}
\usetikzlibrary{calc,backgrounds}


\begin{document}
\begin{preview}
\begin{tikzpicture}[>=latex]


\tikzstyle{state} = [draw, very thick, fill=white, rectangle, minimum height=3em, minimum width=7em, node distance=8em]
\tikzstyle{stateEdgePortion} = [black,very thick];
\tikzstyle{stateEdge} = [stateEdgePortion,->];
\tikzstyle{edgeLabel} = [pos=0.5, text centered, font={\sffamily\small}];


%%%%%%%%%%%%%%%%%%%%%%%%%%%%%%%%%%%%%%%%%%%%%%%%%%%%%%%%%%%%%%%%%%%%%%%%%%%%
%%%%%%%%%%%%%%%%%%%%%%%%%%%%%%%%%%%%%%%%%%%%%%%%%%%%%%%%%%%%%%%%%%%%%%%%%%%%

\node[state, anchor = south](Bprog){
\begin{tabular}{c}
\textsc{\Large B-spline programming}\\[12pt]
\parbox{12cm}{

\textbf{What?}
\begin{itemize}\vspace*{-6pt} \itemsep 0pt
\item optimization problems with tensor-product spline variables \\ ??? overall problem formulation
\item semi-infinite constraints are relaxed to constraints on the B-spline coefficients
\item[$\bm{\hookrightarrow}$] tractable \emph{convex} program (convex B-spline programs)
    \begin{itemize}\vspace*{-6pt}
    \item B-spline conic programs
    \item ???
    \end{itemize}
\item[$\bm{\hookrightarrow}$]\vspace*{-6pt} tractable \emph{non-convex} program (non-convex B-spline programs)
    \begin{itemize}\vspace*{-6pt}
    \item B-spline polynomial programs
    \item ???
    \end{itemize}
\end{itemize}

\textbf{Software}
\begin{itemize}\vspace*{-6pt} \itemsep 0pt
\item Yalmip based Matlab toolbox
\item Python
\end{itemize}

\textbf{Open issues: methodology}
\begin{itemize}\vspace*{-6pt} \itemsep 0pt
\item dual interpretation of the relaxation
\item extension to simplicial splines 
\end{itemize}

\textbf{Open issues: software}
\begin{itemize}\vspace*{-6pt} \itemsep 0pt
\item numerical stability and efficiency, particularly for multi-variate tensor product splines
\item exploiting symmetry and structure
\end{itemize}

}\end{tabular}};


%%%%%%%%%%%%%%%%%%%%%%%%%%%%%%%%%%%%%%%%%%%%%%%%%%%%%%%%%%%%%%%%%%%%%%%%%%%%
%%%%%%%%%%%%%%%%%%%%%%%%%%%%%%%%%%%%%%%%%%%%%%%%%%%%%%%%%%%%%%%%%%%%%%%%%%%%

\node[state, below of = Bprog, anchor = north, node distance = 8cm, xshift = -15cm] (Flat){%
\begin{tabular}{c}
\textsc{\Large Optimal control of flat systems}\\[12pt]
\parbox{12cm}{
\textbf{What?}
\begin{itemize}\vspace*{-6pt} \itemsep 0pt
\item linear and (polynomially ???) nonlinear differentially flat systems
\item flat out parameterized as spline; state and input constraints amount to semi-infinite spline constraints
\item ??? path following / path planning
\item[$\bm{\hookrightarrow}$] \emph{convex} B-spline program
    \begin{itemize}\vspace*{-6pt}
    \item ???
    \end{itemize}
\item[$\bm{\hookrightarrow}$]\vspace*{-6pt} \emph{non-convex} B-spline program
    \begin{itemize}\vspace*{-6pt}
    \item ???
    \end{itemize}
\end{itemize}

\textbf{Software}
\begin{itemize}\vspace*{-6pt} \itemsep 0pt
\item Python
\end{itemize}

\textbf{Open issues: methodology}
\begin{itemize}\vspace*{-6pt} \itemsep 0pt
\item ??? torque constraints for robots
\item ??? constraining trajectories to spline-bounded set
\end{itemize}

\textbf{Open issues: software}
\begin{itemize}\vspace*{-6pt} \itemsep 0pt
\item real-time implementation
\end{itemize}

} \end{tabular} };


\draw ($(Bprog.south)$)
      edge[stateEdge]
      ($(Flat.north)$);
   

%%%%%%%%%%%%%%%%%%%%%%%%%%%%%%%%%%%%%%%%%%%%%%%%%%%%%%%%%%%%%%%%%%%%%%%%%%%%
%%%%%%%%%%%%%%%%%%%%%%%%%%%%%%%%%%%%%%%%%%%%%%%%%%%%%%%%%%%%%%%%%%%%%%%%%%%%

\node[state, below of = Bprog, anchor = north, node distance = 8cm] (Par){%
\begin{tabular}{c}
\textsc{\Large Parametric programming}\\[12pt]
\parbox{12cm}{
\textbf{What?}
\begin{itemize}\vspace*{-6pt} \itemsep 0pt
\item polynomial spline parameterized approximate optimizer functions
\item feasibility for all parameter values amount to semi-infinite spline constraints
\item ??? 
\item[$\bm{\hookrightarrow}$] \emph{convex} B-spline program
    \begin{itemize}\vspace*{-6pt}
    \item ???
    \end{itemize}
\item[$\bm{\hookrightarrow}$]\vspace*{-6pt} \emph{non-convex} B-spline program
    \begin{itemize}\vspace*{-6pt}
    \item ???
    \end{itemize}
\end{itemize}

\textbf{Software}
\begin{itemize}\vspace*{-6pt} \itemsep 0pt
\item Matlab
\end{itemize}

\textbf{Open issues: methodology}
\begin{itemize}\vspace*{-6pt} \itemsep 0pt
\item non-hyperrectangular parameter domains
\item ??? 
\end{itemize}

\textbf{Open issues: software}
\begin{itemize}\vspace*{-6pt} \itemsep 0pt
\item ???
\end{itemize}

} \end{tabular} };


\draw ($(Bprog.south)$)
      edge[stateEdge]
      ($(Par.north)$); 
      

%%%%%%%%%%%%%%%%%%%%%%%%%%%%%%%%%%%%%%%%%%%%%%%%%%%%%%%%%%%%%%%%%%%%%%%%%%%%

\node[state, below of = Par, anchor = north, node distance = 8cm, xshift = -7cm] (TOP2P){%
\begin{tabular}{c}
\textsc{Time-optimal}\\ \textsc{point-to-point motion}
\end{tabular} };

\draw ($(Par.south)$)
      edge[stateEdge]
      ($(TOP2P.north)$); 

\node[state, below of = Par, anchor = north, node distance = 8cm, xshift = -2cm] (EMPC){%
\textsc{Explicit MPC}
};

\draw ($(Par.south)$)
      edge[stateEdge]
      ($(EMPC.north)$); 

\node[state, below of = Par, anchor = north, node distance = 8cm, xshift = 2cm] (LPV){%
\textsc{LPV control}
};

\draw ($(Par.south)$)
      edge[stateEdge]
      ($(LPV.north)$); 

\node[state, below of = Par, anchor = north, node distance = 8cm, xshift = 7cm] (noncvx1){%
\textsc{Nonconvex optimization}
};

\draw ($(Par.south)$)
      edge[stateEdge]
      ($(noncvx1.north)$); 

%%%%%%%%%%%%%%%%%%%%%%%%%%%%%%%%%%%%%%%%%%%%%%%%%%%%%%%%%%%%%%%%%%%%%%%%%%%%
%%%%%%%%%%%%%%%%%%%%%%%%%%%%%%%%%%%%%%%%%%%%%%%%%%%%%%%%%%%%%%%%%%%%%%%%%%%%

\node[state, below of = Bprog, anchor = north, node distance = 8cm, xshift = 15cm] (Pos){%
\begin{tabular}{c}
\textsc{\Large B-spline positivity on spline bounded set}\\[12pt]
\parbox{12cm}{
\textbf{What?}
\begin{itemize}\vspace*{-6pt} \itemsep 0pt
\item tractable certificates for the positivity of a spline a ??? set 
\item[$\bm{\hookrightarrow}$] \emph{sum-of-square} type relaxations 
    \begin{itemize}\vspace*{-6pt}
    \item ???
    \end{itemize}
\item[$\bm{\hookrightarrow}$]\vspace*{-6pt} \emph{Polya} type relaxations
    \begin{itemize}\vspace*{-6pt}
    \item ???
    \end{itemize}
\end{itemize}

\textbf{Software}
\begin{itemize}\vspace*{-6pt} \itemsep 0pt
\item ???
\end{itemize}

\textbf{Open issues: methodology}
\begin{itemize}\vspace*{-6pt} \itemsep 0pt
\item Polya type relaxations
\end{itemize}

\textbf{Open issues: software}
\begin{itemize}\vspace*{-6pt} \itemsep 0pt
\item ???
\end{itemize}
} \end{tabular} };

\draw ($(Bprog.south)$)
      edge[stateEdge]
      ($(Pos.north)$);  
      

%%%%%%%%%%%%%%%%%%%%%%%%%%%%%%%%%%%%%%%%%%%%%%%%%%%%%%%%%%%%%%%%%%%%%%%%%%%%      

\node[state, below of = Pos, anchor = north, node distance = 8cm, xshift = -2cm] (noncvx2){%
\textsc{Nonconvex optimization}
};

\draw ($(Pos.south)$)
      edge[stateEdge]
      ($(noncvx2.north)$); 
      
\node[state, below of = Pos, anchor = north, node distance = 8cm, xshift = 4cm] (rob){%
\begin{tabular}{c}
\textsc{Robust control / }\\ \textsc{robust optimization}
\end{tabular} };

\draw ($(Pos.south)$)
      edge[stateEdge]
      ($(rob.north)$); 
         

\end{tikzpicture}
\end{preview}
\end{document}
